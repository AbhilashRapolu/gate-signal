\iffalse
\let\negmedspace\undefined
\let\negthickspace\undefined
\documentclass[journal,12pt,twocolumn]{IEEEtran}
\usepackage{cite}
\usepackage{amsmath,amssymb,amsfonts,amsthm}
\usepackage{algorithmic}
\usepackage{graphicx}
\usepackage{textcomp}
\usepackage{xcolor}
\usepackage{txfonts}
\usepackage{listings}
\usepackage{enumitem}
\usepackage{mathtools}
\usepackage{gensymb}
\usepackage{comment}
\usepackage[breaklinks=true]{hyperref}
\usepackage{tkz-euclide} 
\usepackage{listings}
\usepackage{gvv}                                        
\def\inputGnumericTable{}                                 
\usepackage[latin1]{inputenc}                         
\usepackage{circuitikz}
\usepackage{color}                                            
\usepackage{array}                                            
\usepackage{longtable}                                       
\usepackage{calc}                                             
\usepackage{multirow} 

\usepackage{hhline}                                           
\usepackage{ifthen}                                           
\usepackage{lscape}


\newtheorem{theorem}{Theorem}[section]
\newtheorem{problem}{Problem}
\newtheorem{proposition}{Proposition}[section]
\newtheorem{lemma}{Lemma}[section]
\newtheorem{corollary}[theorem]{Corollary}
\newtheorem{example}{Example}[section]
\newtheorem{definition}[problem]{Definition}
\newcommand{\BEQA}{\begin{eqnarray}}
\newcommand{\EEQA}{\end{eqnarray}}
\newcommand{\define}{\stackrel{\triangle}{=}}
\theoremstyle{remark}
\newtheorem{rem}{Remark}


\begin{document}
\parindent 0px
\bibliographystyle{IEEEtran}

\title{GATE 2023 - EC 50}
\author{EE23BTECH11220 - R.V.S.S Varun$^{}$% <-this % stops a space
}
\maketitle
\newpage
\bigskip

\renewcommand{\thefigure}{\theenumi}
\renewcommand{\thetable}{\theenumi}
\section*{Question}

Let $x_1\brak{t}$ and $x_2\brak{t}$ be two band-limited signals having bandwidth B = $4\pi\times10^3$
rad/s each. In the figure below, the Nyquist sampling frequency, in
rad/s, required to sample y\brak{t}, is
  \\
  \vspace{25pt}
\begin{figure}[ht]
    \centering
	\begin{circuitikz}
    \begin{circuitikz}
     \draw (0,0) node[left] {$x_1(t)$} to (0.8,0);
     \draw (0,-1) node[left] {$x_2(t)$} to (0.8,-1);
     \draw (1,0) circle(0.2);
     \draw (1,0) node {$\times$};
     \draw (1,-1) circle(0.2);
     \draw (1,-1) node {$\times$};
    \draw (1.2,0) to (1.8,0);
    \draw (1.2,-1) to (1.8,-1);
    \draw (1.8,0) to (1.8,-0.3);
    \draw (1.8,-1) to (1.8,-0.7);
    \draw (1.8,-0.5) circle(0.2);
    \draw (1.8,-0.5) node {$+$};
    \draw[->] (1,0.5) to (1,0.2) ;
    \node at (1,0.6) {cos$(4\pi\times10^3t)$};
    \draw[->] (1,-1.5) to (1,-1.2);
    \node at (1,-1.6) {cos$(12\pi\times10^3t)$};
    \draw (2,-0.5) to (2.3,-0.5);
    \node at (2.6,-0.5) {y(t)};
     
     
\end{circuitikz}

	\end{circuitikz}
    \label{fig:EC50.1}
\end{figure} 

    \brak{a} $20\pi\times10^3$\\
    \brak{b} $40\pi\times10^3$\\
    \brak{c} $8\pi\times10^3$\\
    \brak{d} $32\pi\times10^3$   \hfill(GATE EC 50)\\




\fi

\begin{table}[ht]
    \centering
    \begin{tabular}{|c|c|c|}
    \hline
	Symbol &Description&Value \\
        \hline
	$f_1$&Frequency of cos$\brak{4\pi\times10^3}$&$2\times10^3$ \\
        \hline
	$f_2$&Frequency of cos$\brak{12\pi\times10^3}$&$6\times10^3$ \\
	\hline
	$f_m$&Maximum frequency of the output signal&- \\
	\hline
	 $\omega_{m}$&-&$2\pi f_m$ \\
         \hline
	 $\omega_{s}$&Nyquist sampling rate&$2\omega_m$ \\
         \hline
    \end{tabular}
 
    \caption{Table of parameters}
    \label{tab:EC50.1}
\end{table}


\begin{figure}[ht]
    \centering
	
    \begin{circuitikz}
    \draw[->] (-5,0) to (5,0);
    \draw[->] (0,-1) to (0,5);
    \draw (-3,0) to (0,3);
    \draw (0,3) to (3,0);
    \draw (-3,-0.5) node {$-2\times10^3$};
    \draw (3,-0.5) node {$2\times10^3$};
    \draw (0,5.5) node {$X_{1}\brak{f}$};

\end{circuitikz}

	
    \label{fig:EC50.2}
\end{figure} 


\begin{figure}[ht]
    \centering
	
    \begin{circuitikz}
    \draw[->] (-5,0) to (5,0);
    \draw[->] (0,-1) to (0,5);
    \draw (-3,0) to (0,3);
    \draw (0,3) to (3,0);
    \draw (-3,-0.5) node {$-2\times10^3$};
    \draw (3,-0.5) node {$2\times10^3$};
    \draw (0,5.5) node {$X_{2}\brak{f}$};
\end{circuitikz}

	
    \label{fig:EC50.3}
\end{figure} 

 From question figure ,
\begin{align}
y\brak{t}=x_1\brak{t}\times cos\brak{4\pi\times10^3t}+x_2\brak{t}\times cos\brak{12\pi\times10^3t}
\end{align}
\begin{align}
	y\brak{t}=y_1\brak{t}+y_2\brak{t}\\
	Y\brak{f}=Y_1\brak{f}+Y_2\brak{f}\label{EC50.1}
\end{align}
\begin{align}
	Y_1\brak{f}&=X_1\brak{f}*\frac{1}{2}[\delta\brak{f-f_1}+\delta\brak{f+f_1}]\\
	      &=\frac{1}{2}[X_1\brak{f-f_1}+X_1\brak{f+f_1}]
\end{align}
\begin{figure}[ht]
    \centering
	
    \begin{circuitikz}[scale=0.8]
    \draw[->] (-5,0) to (5,0);
     \draw[->] (0,-1) to (0,4);
     \draw (-2,0) to (-1,2);
     \draw (-1,2) to (0,0);
     \draw (0,0) to (1,2);
     \draw (1,2) to (2,0);
     \draw (-2,-0.5) node {$-4\times10^3$};
     \draw (2,-0.5) node {$4\times10^3$};
    \draw (0,4.5) node {$Y_{1}\brak{f}$};
\end{circuitikz}

	
    \label{fig:EC50.4}
\end{figure} 

\begin{align}
	Y_2\brak{f}&=X_2\brak{f}*\frac{1}{2}[\delta\brak{f-f_2}+\delta\brak{f+f_2}]\\ 
	&=\frac{1}{2}[X_2\brak{f-f_2}+X_2\brak{f+f_2}]
\end{align}
\clearpage

\begin{figure}[ht]
    \centering
	
    \begin{circuitikz}[scale=0.8]
    \draw[->] (-5,0) to (5,0);
     \draw[->] (0,-1) to (0,4);
     \draw (-4,0) to (-3,2);
     \draw (-3,2) to (-2,0);
     \draw (2,0) to (3,2);
     \draw (3,2) to (4,0);
     \draw (-4,-0.5) node {$-8\times10^3$};
     \draw (-2,-0.5) node {$-4\times10^3$};
     \draw (2,-0.5) node {$-4\times10^3$};
     \draw (4,-0.5) node {$-8\times10^3$};
    \draw (0,4.5) node {$Y_{2}\brak{f}$};
\end{circuitikz}

	
    \label{fig:EC50.5}
\end{figure} 

From \eqref{EC50.1}:
\begin{figure}[ht]
    \centering
	
    \begin{circuitikz}
    \draw[->] (-5,0) to (5,0);
    \draw[->] (0,-1) to (0,5);
    \draw (-4,0) to (-3,2);
    \draw (-3,2) to (-2,0);
    \draw (-2,0) to (-1,2);
    \draw (-1,2) to (0,0);
    \draw (0,0) to (1,2);
    \draw (1,2) to (2,0);
    \draw (2,0) to (3,2);
    \draw (3,2) to (4,0);
    \draw (-2,-0.5) node {$-4\times10^3$};
    \draw (2,-0.5) node {$4\times10^3$};
    \draw (4,-0.5) node {$8\times10^3$};
    \draw (-4,-0.5) node {$-8\times10^3$};
    \draw (0,5.5) node {$Y\brak{f}$};
\end{circuitikz}

	
    \label{fig:EC50.6}
\end{figure} 

From table,
\begin{align}
\omega_{m}=16\pi\times10^3 rad/sec.
\end{align}
\begin{align}
\omega_{s}=2\omega_{m}=32\pi\times10^3 rad/sec.
\end{align}

